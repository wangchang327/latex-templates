\documentclass[11pt]{article}
\usepackage{snippets/font-setup}
\usepackage{snippets/title-setup}
\usepackage{snippets/math-setup}
\usepackage{snippets/geometry-setup}

\title{\textbf{作业 1}}
\author{张三 \\ 114514}
\date{2022 年春}

\begin{document}
    \maketitle

    \begin{hwsol}
        在级数展开中作用 $\hat H$ 的时间表达式得
        \[ \hat Hu=i \hbar \partial_t \sum_{n=1}^\infty c_n(t) \Psi_n(x)=i \hbar \sum_{n=1}^\infty c_n'(t) \Psi_n(x). \]
        又作用 $\hat H$ 的 $x$ 的表达式得
        \[ \hat H u = \sum_{n=1}^\infty c_n(t)E_n \Psi_n(x). \]
        
        因为特征函数是线性无关的, 所以 $i \hbar c'_n(t)=c_n(t)E_n$, 解得 $c_n(t) = \exp \left(\frac{E_nt}{i \hbar}\right)c_n(0)$. 对于 $t=0$ 时,
        \[ u(x, 0)=u_0(x) = \sum_{n=1}^\infty c_n(0) \Psi_n(x), \]
        两边乘以 $\Psi_m(x)$ 并利用正交性
        \[ \int_{\mathbb R} u_0(x) \Psi_m(x)\,\mathrm dx=c_m(0). \]
        代入前面 $c_n(t)$ 的表达式就得到了题目要求的式子.
        
        求 Green 函数则如下运算:
        \begin{align*}
            u(x, t) &= \sum_{n=1}^\infty \Psi_n(x) \exp \left(\frac{E_nt}{i \hbar}\right) \int_{\mathbb R} u_0(y) \Psi_n(y)\,\mathrm dy = \sum_{n=1}^\infty \int_{\mathbb R} u_0(y) \Psi_n(x) \exp \left(\frac{E_nt}{i \hbar}\right) \Psi_n(y)\,\mathrm dy
            \\&= \int_{\mathbb R} \sum_{n=1}^\infty u_0(y) \Psi_n(x) \exp \left(\frac{E_nt}{i \hbar}\right) \Psi_n(y)\,\mathrm dy = \int_{\mathbb R} u_0(y) \sum_{n=1}^\infty \Psi_n(x) \exp \left(\frac{E_nt}{i \hbar}\right) \Psi_n(y)\,\mathrm dy.
        \end{align*}
        所以
        \[ G(x, x', t) = \sum_{n=1}^\infty \Psi_n(x) \Psi_n(x') \exp \left(\frac{E_nt}{i \hbar} \right). \]
    \end{hwsol}
\end{document}