% Copyright 2020 王畅 (WANG Chang).
% Permission is granted to copy, distribute and/or modify this
% document under the terms of the Creative Commons
% Attribution 4.0 International (CC BY 4.0)
% http://creativecommons.org/licenses/by/4.0/

\documentclass{exam}
\usepackage{ctex}
\usepackage{amsmath}
\usepackage{amssymb}
\usepackage{amsthm}
\usepackage{mathrsfs} % 花写字母
\usepackage{xcolor}
\usepackage{paralist} % 扩展列表, 如紧凑列表
\usepackage{mathtools} % 一些额外符号
\usepackage{booktabs} % fancy表格
\usepackage{graphicx} % 图片和浮动
\usepackage{float}
\pointname{分}

% 页眉页脚
\pagestyle{headandfoot}
\firstpageheadrule
\firstpageheader{00000000: 肥宅导论}{作业 \#1}{2333/02/33} % 修改为作业布置的日期
\firstpagefooter{}{第 \thepage 页,共 \numpages 页}{}
\runningfooter{}{第 \thepage 页,共 \numpages 页}{}

\renewcommand{\solutiontitle}{\noindent{\heiti 解答:}\enspace} % 答案前面的提示词
\theoremstyle{remark}
\newtheorem*{remark}{评论}

\usepackage{tikz}
\usetikzlibrary{shapes.symbols}
\newenvironment{hint}{% 提示词绘制
    \ifvmode
        \ignorespaces
    \else
        \quad
    \fi
    \begin{tikzpicture}[baseline=(H.base), every node/.style={signal, draw, very thin, signal to=east, signal from=nowhere, signal pointer angle=120, inner sep=2pt}]
        \node[anchor=mid west] (H) at (0,0) {\heiti\footnotesize 提示};
    \end{tikzpicture}
}{}

% 字体更换
\usepackage{newtxtext}
\usepackage{newtxmath}
\DeclareSymbolFont{CMlargesymbols}{OMX}{cmex}{m}{n}
\let\sumop\relax\let\prodop\relax
\DeclareMathSymbol{\sumop}{\mathop}{CMlargesymbols}{"50}
\DeclareMathSymbol{\prodop}{\mathop}{CMlargesymbols}{"51}

\begin{document}
    \vspace*{0in}
    \begin{center}
        \Huge\bf 作业 \#1 % 修改为作业编号
    \end{center}
    
    \unframedsolutions % 取消答案的框
    \SolutionEmphasis{\fangsong} % 答案字体设置
    \printanswers % 打印答案, 布置作业时请注释掉
    
    {\small\kaishu 说明: 本作业不用交, 交了记0分.}

    % \question表示一个题
    % part和subpart用来分小问
    % solution用来写答案
    \begin{questions}
        \question[5] 解释你一天是如何达成肥宅目标的.
        \begin{solution}
            (留白)
        \end{solution}
        \begin{remark}
            留白, 充分显示了肥宅的本性.
        \end{remark}

        \question[5] 考虑算式 $1 + 2$.
        \begin{parts}
            \part[2] 请问它等于什么? \begin{hint} 答案不是3. \end{hint}
            \part[3] 解释你的答案.
        \end{parts}
        \begin{solution}
            \begin{parts}
                \part 答案为 $2 + 1$.
                \part 因为自然数的加法做成一个 Abel 群.
            \end{parts}
        \end{solution}
    \end{questions}
\end{document}